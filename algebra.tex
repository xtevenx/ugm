\documentclass{scrartcl}

\KOMAoptions{parskip=half}

\usepackage{amsmath}
\usepackage{amssymb}
\usepackage{amsthm}
\usepackage{hyperref}

\theoremstyle{definition}
\newtheorem{definition}{Definition}[section]

\theoremstyle{plain}
\newtheorem{lemma}{Lemma}[section]
\newtheorem{proposition}{Proposition}[section]
\newtheorem{theorem}{Theorem}[section]
\newtheorem{corollary}{Corollary}[theorem]

% \theoremstyle{remark}
\newtheorem{remark}{Remark}[section]

\begin{document}

\section{Rings and Fields}

\subsection{Introduction}

\begin{definition}[Binary Operation]
    \label{def:binary operation}

    \newcommand{\Q}{\mathbb{Q}}
    \newcommand{\R}{\mathbb{R}}
    \newcommand{\Z}{\mathbb{Z}}

    An \textbf{internal binary operation} defined on a set $S$ is a mapping $f:S\times S\to S$.

    The following are examples of binary operations:
    \begin{enumerate}
        \item Addition on the set of nonnegative integers $\Z_{\geq 0}$.
        \item Subtraction on the set of integers $\Z$.
        \item Multiplication on the set of $n\times n$ matrices with real number entries $\R^{n\times n}$.
    \end{enumerate}

    The following are not binary operations:
    \begin{enumerate}
        \item Subtraction on the set of nonnegative integers $\Z_{\geq 0}$, because $0-1=-1$ is not in $\Z_{\geq 0}$.
        \item Division on the set of rational numbers $\Q$, because $1/0$ is not defined.
    \end{enumerate}

    In this document, we will refer to these simply as \textbf{binary operations}.
\end{definition}

\begin{definition}[Ring]
    \label{def:ring}

    \newcommand{\Z}{\mathbb{Z}}

    A \textbf{ring} is a set $R$ with two \hyperref[def:binary operation]{binary operations}---addition $(+)$ and
    multiplication $(\cdot)$---satisfying the following \textbf{ring axioms}:
    \begin{enumerate}
        \item Addition is associative, meaning $(a+b)+c=a+(b+c)$ for all $a,b,c\in R$.
        \item Addition is commutative, meaning $a+b=b+a$ for all $a,b\in R$.

        \item
            There is an identity element with respect to addition, meaning there exists $0\in R$ satisfying $a+0=a$ and
            $0+a=a$ for all $a\in R$.

        \item
            Every element has an inverse with respect to addition, meaning that for all $a\in R$, there exists $b\in R$
            satisfying $a+b=0$ and $b+a=0$.

        \item Multiplication is associative, meaning $(a\cdot b)\cdot c=a\cdot (b\cdot c)$ for all $a,b,c\in R$.

        \item
            There is an identity element with respect to multiplication, meaning there exists $1\in R$ satisfying
            $a\cdot 1=a$ and $1\cdot a=a$ for all $a\in R$.

        \item
            Multiplication is distributive over addition, meaning $a\cdot (b+c)=a\cdot b+a\cdot c$ and
            $(b+c)\cdot a=b\cdot a+c\cdot a$ for all $a,b,c\in R$.
    \end{enumerate}

    The following are examples of rings.
    \begin{enumerate}
        \item The set of integers $\Z$ with addition and multiplication defined conventionally.
        \item
            The set of $n\times n$ matrices with entries from a ring $R$, along with the addition and multiplication
            operations copied from $R$.
    \end{enumerate}

    By convention, the addition operation of a ring is denoted $(+)$, and the multiplication operation of a ring is
    denoted $(\cdot)$.
    Furthermore, adding an additive inverse is usually shortened to ``subtraction'', meaning $a+(-b)\equiv a-b$, and
    the multiplication operation is usually shortened to be written without the $(\cdot)$, meaning $a\cdot b\equiv ab$.
    We will adopt all of these these conventions.

    % pandoc does not properly process enumeration references, otherwise we could label the item and reference that,
    % instead of hard referencing 'condition (6)'.
    Some texts do not require condition (6) to hold, that is, their rings do not necessarily an identity element with
    respect to multiplication.
    See this \href{https://en.wikipedia.org/wiki/Ring_(mathematics)#Notes_on_the_definition}{Wikipedia article} for
    more information.
    As a subjective opinion, this seems to be somewhat of a stylistic choice: if there is no multiplicative identity,
    then we cannot take the product of an arbitrary collection of elements, because this would be undefined if the
    collection is empty.
    On the other hand, sometimes it simply does not make sense to force the existence of a multiplicative identity.
    This may not be the best example, but later we will define a special type of ``subring'' called an ideal, which
    should not be defined as requiring a multiplicative identity.
\end{definition}

\begin{definition}[Commutative ring]
    \label{def:commutative ring}

    A \textbf{commutative ring} is a \hyperref[def:ring]{ring} $R$ with the additional property that the multiplication
    operation is commutative, meaning $ab=ba$ for all $a,b\in R$.
\end{definition}

\begin{proposition}
    \label{prop:unique additive identity}
    Let $R$ be a \hyperref[def:ring]{ring}, then $R$ has exactly one additive identity.
\end{proposition}

\begin{proof}
    By definition, we know that $R$ has at least one additive identity, so we will show it has at most one.
    Let $r,s\in R$ be two additive identities, and see that $r+s=s$ because $r$ is an additive identity.
    But also, we see that $r+s=r$, because $s$ is an additive identity, so it follows that $s=r$.
    This means every additive identity is the same, so there is at most one.
\end{proof}

\begin{proposition}
    \label{prop:unique multiplicative identity}
    Let $R$ be a \hyperref[def:ring]{ring}, then $R$ has exactly one multiplicative identity.
\end{proposition}

\begin{proof}
    Refer to the proof of the \hyperref[prop:unique additive identity]{uniqueness of the additive identity}, and
    replace addition with multiplication.
\end{proof}

\begin{remark}
    Since we now know that rings contain exactly one additive and multiplicative identity, we can (and will) denote
    these values with $0$ and $1$ respectively.
\end{remark}

\begin{proposition}
    \label{prop:unique additive inverse}
    Let $R$ be a \hyperref[def:ring]{ring}, and $r\in R$.
    Then $r$ has exactly one additive inverse.
\end{proposition}

\begin{proof}
    By definition, we know that $r$ has at least one additive inverse, so we show it has at most one.
    Suppose $s,t\in R$ are both additive inverses of $r$, and see the following:
    \begin{align}
        r+s=0\text{ and }r+t=0
        \implies& s+(r+s)=s+(r+t) \\
        \iff& (s+r)+s=(s+r)+t \\
        \iff& 0+s=0+t \\
        \iff& s=t
    \end{align}
    This means every additive inverse of $r$ is the same, so there is at most one.
\end{proof}

\begin{remark}
    Let $R$ be a \hyperref[def:ring]{ring}, and $r\in R$.
    Since we now know the additive inverse of $r$ is unique, we can (and will) denote this value with $-r$.
\end{remark}

\begin{proposition}
    \label{prop:multiply by zero}
    Let $R$ be a \hyperref[def:ring]{ring}, and $r\in R$.
    Then $0r=0$.
\end{proposition}

\begin{proof}
    See that
    \begin{align}
        (0+0)r=0r
        \iff& 0r+0r=0r \\
        \iff& 0r+0r-0r=0r-0r \\
        \iff& 0r=0
    \end{align}
\end{proof}

\begin{proposition}
    Let $R$ be a \hyperref[def:ring]{ring}, and $r,s\in $.
    Then $(-r)s=-(rs)=r(-s)$.
\end{proposition}

\begin{proof}
    For the first equality, we use \hyperref[prop:multiply by zero]{multiplication by zero} to see that
    \begin{align}
        0s=0
        \iff& ((-r)+r)s=0 \\
        \iff& (-r)s+rs=0 \\
        \iff& (-r)s+rs-(rs)=0-(rs) \\
        \iff& (-r)s=-(rs)
    \end{align}
    For the second equality, we use the same approach to see that
    \begin{align}
        r0=0
        \iff& r((-s)+s)=0 \\
        \iff& r(-s)+rs=0 \\
        \iff& r(-s)+rs-(rs)=0-(rs) \\
        \iff& r(-s)=-(rs)
    \end{align}
\end{proof}

\end{document}
