\documentclass{scrartcl}

\usepackage{template}

\begin{document}
\section{Rings and Fields}

\subsection{Introduction}

\begin{definition}[Binary Operation]
    \label{def:binary operation}
    \newcommand{\Q}{\mathbb{Q}}
    \newcommand{\R}{\mathbb{R}}
    \newcommand{\Z}{\mathbb{Z}}
    An \textbf{internal binary operation} defined on a set $S$ is a mapping $f:S\times S\to S$.

    In this document, we will refer to these simply as \textbf{binary operations}.

    The following are examples of binary operations:
    \begin{enumerate}
        \item Addition on the set of nonnegative integers $\Z_{\geq 0}$.
        \item Subtraction on the set of integers $\Z$.
        \item Multiplication on the set of $n\times n$ matrices with real number entries $\R^{n\times n}$.
    \end{enumerate}

    The following are not binary operations:
    \begin{enumerate}
        \item Subtraction on the set of nonnegative integers $\Z_{\geq 0}$, because $0-1=-1$ is not in $\Z_{\geq 0}$.
        \item Division on the set of rational numbers $\Q$, because $1/0$ is not defined.
    \end{enumerate}
\end{definition}

\begin{definition}[Ring]
    \label{def:ring}
    \newcommand{\Z}{\mathbb{Z}}
    A \textbf{ring} is a set $R$ with two \hyperref[def:binary operation]{binary operations}---addition $(+)$ and
    multiplication $(\cdot)$---satisfying the following \textbf{ring axioms}:
    \begin{enumerate}
        \item Addition is associative, meaning $(a+b)+c=a+(b+c)$ for all $a,b,c\in R$.
        \item Addition is commutative, meaning $a+b=b+a$ for all $a,b\in R$.

        \item
            There is an identity element with respect to addition, meaning there exists $0\in R$ satisfying $a+0=a$ and
            $0+a=a$ for all $a\in R$.

        \item
            Every element has an inverse with respect to addition, meaning that for all $a\in R$, there exists $b\in R$
            satisfying $a+b=0$ and $b+a=0$.

        \item Multiplication is associative, meaning $(a\cdot b)\cdot c=a\cdot (b\cdot c)$ for all $a,b,c\in R$.

        \item
            There is an identity element with respect to multiplication, meaning there exists $1\in R$ satisfying
            $a\cdot 1=a$ and $1\cdot a=a$ for all $a\in R$.

        \item
            Multiplication is distributive over addition, meaning $a\cdot (b+c)=a\cdot b+a\cdot c$ and
            $(b+c)\cdot a=b\cdot a+c\cdot a$ for all $a,b,c\in R$.
    \end{enumerate}

    The following are examples of rings.
    \begin{enumerate}
        \item The set of integers $\Z$ with addition and multiplication defined conventionally.
        \item
            The set of $n\times n$ matrices with entries from a ring $R$, along with the addition and multiplication
            operations copied from $R$.
    \end{enumerate}

    By convention, the addition operation of a ring is denoted $(+)$, and the multiplication operation of a ring is
    denoted $(\cdot)$.
    Furthermore, adding an additive inverse is usually written as ``subtraction'', meaning $a+(-b)\equiv a-b$, and the
    multiplication operation is usually written without $(\cdot)$, meaning $a\cdot b\equiv ab$.
    We will adopt all of these these conventions.

    % NOTE: pandoc does not properly process enumeration references, otherwise we could label the item and reference
    % that, instead of hard referencing 'condition (6)'.
    Some texts do not require condition (6) to hold, that is, their rings do not necessarily an identity element with
    respect to multiplication.
    See this \href{https://en.wikipedia.org/wiki/Ring_(mathematics)#Notes_on_the_definition}{Wikipedia article} for
    more information.
    As a subjective opinion, this seems to be somewhat of a stylistic choice: if there is no multiplicative identity,
    then we cannot take the product of an arbitrary collection of elements, because this would be undefined if the
    collection is empty.
    On the other hand, sometimes it simply does not make sense to force the existence of a multiplicative identity.
    This may not be the best example, but later we will define a special type of ``subring'' called an ideal, which
    should not be defined as requiring a multiplicative identity.
\end{definition}

\begin{definition}[Commutative ring]
    \label{def:commutative ring}
    A \textbf{commutative ring} is a \hyperref[def:ring]{ring} $R$ with the additional property that the multiplication
    operation is commutative, meaning $ab=ba$ for all $a,b\in R$.
\end{definition}

\begin{proposition}
    \label{prop:unique additive identity}
    Let $R$ be a \hyperref[def:ring]{ring}, then $R$ has exactly one additive identity.
\end{proposition}

\begin{proof}
    By definition, we know that $R$ has at least one additive identity, so we will show it has at most one.
    Let $a,b\in R$ be two additive identities, and see that $a+b=b$ because $a$ is an additive identity.
    But also, we see that $a+b=a$, because $b$ is an additive identity, so it follows that $a=b$.
    This means every additive identity is the same, so there is at most one.
\end{proof}

\begin{proposition}
    \label{prop:unique multiplicative identity}
    Let $R$ be a \hyperref[def:ring]{ring}, then $R$ has exactly one multiplicative identity.
\end{proposition}

\begin{proof}
    Refer to the proof of the \hyperref[prop:unique additive identity]{uniqueness of the additive identity}, and
    replace addition with multiplication.
\end{proof}

\begin{remark}
    Since we now know that rings contain exactly one additive and multiplicative identity, we can (and will) denote
    these values with $0$ and $1$ respectively.
\end{remark}

\begin{proposition}
    \label{prop:unique additive inverse}
    Let $R$ be a \hyperref[def:ring]{ring}, and $a\in R$.
    Then $a$ has exactly one additive inverse.
\end{proposition}

\begin{proof}
    By definition, we know that $a$ has at least one additive inverse, so we show it has at most one.
    Suppose $b,c\in R$ are both additive inverses of $a$, and see the following:
    \begin{align}
        a+b=0\text{ and }a+c=0
        \implies& b+(a+b)=b+(a+c) \\
        \iff& (b+a)+b=(b+a)+c \\
        \iff& 0+b=0+c \\
        \iff& b=c
    \end{align}
    This means every additive inverse of $a$ is the same, so there is at most one.
\end{proof}

\begin{remark}
    Let $R$ be a \hyperref[def:ring]{ring}, and $a\in R$.
    Since we now know the additive inverse of $a$ is unique, we can (and will) denote this value with $-a$.
\end{remark}

\begin{proposition}
    \label{prop:multiply by zero}
    Let $R$ be a \hyperref[def:ring]{ring}, and $a\in R$.
    Then $0a=0$.
\end{proposition}

\begin{proof}
    See that
    \begin{align}
        (0+0)a=0a
        \iff& 0a+0a=0a \\
        \iff& 0a+0a-(0a)=0a-(0a) \\
        \iff& 0a=0
    \end{align}
\end{proof}

\begin{proposition}
    Let $R$ be a \hyperref[def:ring]{ring}, and $a,b\in R$.
    Then $(-a)b=-(ab)=a(-b)$.
\end{proposition}

\begin{proof}
    For the first equality, we use \hyperref[prop:multiply by zero]{multiplication by zero} to see that
    \begin{align}
        0b=0
        \iff& ((-a)+a)b=0 \\
        \iff& (-a)b+ab=0 \\
        \iff& (-a)b+ab-(ab)=0-(ab) \\
        \iff& (-a)b=-(ab)
    \end{align}
    For the second equality, we use the same approach to see that
    \begin{align}
        a0=0
        \iff& a((-b)+b)=0 \\
        \iff& a(-b)+ab=0 \\
        \iff& a(-b)+ab-(ab)=0-(ab) \\
        \iff& a(-b)=-(ab)
    \end{align}
\end{proof}

\begin{remark}
    % WARNING: this reference the 'last two proofs', which may not be here if we rearrange the document.
    From these last two proofs, we see that it is allowed to ``cancel out'' identical addition terms from both sides of
    an equation.
    This is due to the associativity and commutativity of addition, along with the existence of an additive identity
    for every element.
    Keep this in mind as, in the future, we may not explicitly write the subtraction on both sides.
\end{remark}

\begin{definition}[Zero divisor]
    \label{def:zero divisor}
    Let $R$ be a \hyperref[def:ring]{ring}, and $a\in R$.
    Then $a$ is a \textbf{left zero divisor} if there exists nonzero $b\in R$ such that $ab=0$, a
    \textbf{right zero divisor} if there exists nonzero $c\in R$ such that $ca=0$, and a \textbf{zero divisor} if it is
    either a left zero divisor or a right zero divisor.
\end{definition}

\begin{proposition}
    Let $R$ be a \hyperref[def:ring]{ring}, and $a\in R$ nonzero.
    Then $a$ is a \hyperref[def:zero divisor]{left zero divisor} if and only if there exists $b,c\in R$ satisfying
    $b\neq c$ and $ab=ac$.
\end{proposition}

\begin{proof}
    Suppose $a$ is a left zero divisor, so there exists nonzero $b\in R$ such that $ab=0$.
    Set $c=0$ and see that $b\neq c$ and $ab=ac$.

    Suppose there exists $b,c\in R$ satisfying $b\neq c$ and $ab=ac$.
    We see that $ab=ac\iff ab-ac=0\iff a(b-c)=0$, but since $b\neq c\iff b-c\neq 0$, this means $a$ is a left zero
    divisor.
\end{proof}

\begin{definition}[Integral domain]
    An \textbf{integral domain} is a \hyperref[def:commutative ring]{commutative ring} $R$ with the additional property
    that $ab=0\implies a=0\text{ or }b=0$ for all $a,b\in R$.
    This is equivalent to saying that $R$ has no nonzero \hyperref[def:zero divisor]{zero divisors}.
\end{definition}

\begin{definition}[Unit]
    \label{def:unit}
    Let $R$ be a \hyperref[def:ring]{ring}, and $a\in R$.
    Then $a$ is a \textbf{unit} if there exists $b\in R$ such that $ab=1$ and $ba=1$.
\end{definition}

\begin{proposition}
    \label{prop:unique multiplicative inverse}
    Let $R$ be a \hyperref[def:ring]{ring}, and $a\in R$ a unit.
    Then $a$ has exactly one multiplicative inverse.
\end{proposition}

\begin{proof}
    Refer to the proof of the \hyperref[prop:unique additive inverse]{uniqueness of the additive inverse}, and replace
    addition with multiplication and $0$ with $1$.
\end{proof}

\begin{remark}
    Let $R$ be a \hyperref[def:ring]{ring}, and $a\in R$ a unit.
    Since we now know the multiplicative inverse of $a$ is unique, we can (and will) denote this value with $a^{-1}$.
\end{remark}

\begin{proposition}
    Let $R$ be a \hyperref[def:ring]{ring}, and $a\in R$.
    Then $a$ cannot be both a \hyperref[def:unit]{unit} and a \hyperref[def:zero divisor]{zero divisor}.
\end{proposition}

\begin{proof}
    Towards a contradiction, suppose $a$ is both a unit and a zero divisor.
    Without loss of generality, assume $a$ is a left zero divisor, so there exists nonzero $b\in R$ such that $ab=0$.
    Since $a$ is a unit, there exists $c\in R$ such that $ca=1$.
    Now we see that $ab=0\iff cab=c0\iff b=0$, but $b$ is nonzero, a contradiction.
\end{proof}

\begin{definition}[Zero ring]
    \label{def:nonzero ring}
    A \textbf{zero ring} is a \hyperref[def:ring]{ring} containing a single element.
    This kind of ring is ``zero'' because the single element must be the additive identity, denoted by $0$.
\end{definition}

\begin{definition}[Field]
    \label{def:field}
    A \textbf{field} is a \hyperref[def:nonzero ring]{nonzero} \hyperref[def:commutative ring]{commutative ring} $R$
    with the additional property that every nonzero element is a \hyperref[def:unit]{unit}.
\end{definition}

\begin{definition}[Characteristic]
    \label{def:characteristic}
    The \textbf{characteristic} of a \hyperref[def:ring]{ring} $R$ is the least positive integer $c$ satisfying
    $\underbrace{1+\dots+1}_{c\text{ times}}=0$.
    If such an integer does not exist, then the characteristic is $0$.
\end{definition}

\begin{proposition}
    Let $R$ be a \hyperref[def:ring]{ring} with \hyperref[def:characteristic]{characteristic} $c$.
    If $c$ is a composite number, then $R$ has \hyperref[def:zero divisor]{zero divisors}.
\end{proposition}

\begin{proof}
    Since $c$ is composite, there exists $m,n\in\mathbb{Z}_{\geq 0}$ satisfying $c=mn$ and $m,n<c$.
    Let $r=\underbrace{1+\dots+1}_{m\text{ times}}\in R$ and $s=\underbrace{1+\dots+1}_{n\text{ times}}\in R$, which
    are both nonzero by minimality of $c$.
    Now distributivity gives us
    \[rs=(\underbrace{1+\dots+1}_{m\text{ times}})(\underbrace{1+\dots+1}_{n\text{ times}})=\underbrace{1+\dots+1}_{c\text{ times}}=0,\]
    so $r$ and $s$ are zero divisors in $R$.
\end{proof}

\begin{theorem}[Euler]
    Let $R$ be a \hyperref[def:commutative ring]{commutative ring}, and $R^\times$ the set of
    \hyperref[def:unit]{units} in $R$.
    If $n\coloneqq\lvert R^\times\rvert$ is finite, then $a^n=1$ for all $a\in R^\times$.
\end{theorem}

\begin{proof}
    Let $\phi:R^\times\to R^\times$ be a mapping defined by $r\mapsto ar$.
    Since $\phi$ is a mapping between equally sized finite sets, we will show that it is injective, which would imply
    that it is bijective.
    See that $\phi(r)=\phi(s)\iff ar=as\iff (a^{-1})ar=(a^{-1})as\iff r=s$.

    Now we know that $\phi$ is a bijection, which means $R^\times=\phi(R^\times)$.
    The theorem follows from the following equality of products:
    \begin{align}
        \prod_{r\in R^\times}\phi(r)=\prod_{r\in R^\times}r
        \iff&\prod_{r\in R^\times}ar=\prod_{r\in R^\times}r \\
        \iff&a^n\prod_{r\in R^\times}r=\prod_{r\in R^\times}r \\
        \iff&a^n=1
    \end{align}
\end{proof}

\begin{remark}
    We note that commutativity of multiplication is required for this proof, otherwise we couldn't pull out a factor of
    $a$ from each factor of the left hand side.
    However Euler's theorem also holds for noncommutative rings.
\end{remark}

\end{document}
