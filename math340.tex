\documentclass{scrartcl}

\usepackage{amssymb}
\usepackage{amsthm}
\usepackage{hyperref}

\theoremstyle{definition}
\newtheorem{definition}{Definition}[section]

\theoremstyle{plain}
\newtheorem{lemma}{Lemma}[section]
\newtheorem{proposition}{Proposition}[section]
\newtheorem{theorem}{Theorem}[section]
\newtheorem{corollary}{Corollary}[theorem]

\theoremstyle{remark}
\newtheorem{remark}{Remark}[section]

\begin{document}

\begin{definition}[Binary Operation]
    \label{def:binary operation}

    \newcommand{\C}{\mathbb{C}}
    \newcommand{\Q}{\mathbb{Q}}
    \newcommand{\Z}{\mathbb{Z}}

    An \textbf{internal binary operation} defined on a set $S$ is a mapping $f:S\times S\to S$.
    In this document, we will refer to this simply as a \textbf{binary operation}.

    The following are examples of binary operations:
    \begin{enumerate}
        \item Addition on the set $\Z$.
        \item Subtraction on the set $\Q$.
        \item Multiplication on the set $\C^{n\times n}$ (the set of $n\times n$ matrices in the complex numbers).
    \end{enumerate}

    The following are not binary operations:
    \begin{enumerate}
        \item Subtraction on the set $\Z_{\geq 0}$ (because $0-1=-1$ is not in $\Z_{\geq 0}$).
        \item Division on the set $\Q$ (because $1/0$ is not defined).
    \end{enumerate}
\end{definition}

\begin{definition}[Ring]
    \label{def:ring}

    A \textbf{ring} is a set $R$ with two \hyperref[def:binary operation]{binary operations}---addition $(+)$ and
    multiplication $(\cdot)$---satisfying the following \textbf{ring axioms}:
    \begin{enumerate}
        \item Addition is associative, meaning $(a+b)+c=a+(b+c)$ for all $a,b,c\in R$.
        \item Addition is commutative, meaning $a+b=b+a$ for all $a,b\in R$.

        \item
            There is an identity element with respect to addition, meaning there exists $0\in R$ satisfying $a+0=a$ and
            $0+a=a$ for all $a\in R$.

        \item
            Every element has an inverse with respect to addition, meaning that for all $a\in R$, there exists $b\in R$
            satisfying $a+b=0$.

        \item Multiplication is associative, meaning $(a\cdot b)\cdot c=a\cdot (b\cdot c)$ for all $a,b,c\in R$.

        \item
            There is an identity element with respect to multiplication, meaning there exists $1\in R$ satisfying
            $a\cdot 1=a$ and $1\cdot a=a$ for all $a\in R$.

        \item
            Multiplication is left-distributive over addition, meaning $a\cdot (b+c)=a\cdot b+a\cdot c$ for all
            $a,b,c\in R$.

        \item
            Multiplication is right-distributive over addition, meaning $(b+c)\cdot a=b\cdot a+c\cdot a$ for all
            $a,b,c\in R$.
    \end{enumerate}
\end{definition}

\end{document}
